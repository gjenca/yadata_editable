\documentclass{amsart}
\newtheorem{proposition}{Proposition}
\theoremstyle{definition}
\newtheorem{definition}{Definition}
\begin{document}
\begin{definition}
\label{def:minimal}
Let $x\in B(M)$.
We say that an $M$-chain representation $(x_i)_{i=1}^{2n}$ 
of $x$ is minimal iff
\begin{enumerate}
\item[(a)] For all $2\leq i\leq n$, 
$(x_{2i}\ominus x_{2i-1})\wedge(x_{2i-1}\ominus x_{2i-2})\neq0$ and,
\item[(b)] if $n>0$, $(x_2\ominus x_1)\wedge x_1\neq 0$.
\end{enumerate}
\end{definition}
\begin{proposition}
Every $x\in B(M)$ has a minimal $M$-chain representation.
\end{proposition}
\begin{proof}
Suppose that $x$ has an $M$-chain representation $(x_i)_{i=1}^{2n}$.
We shall apply induction with respect to $n$. If $n=0$, there is nothing to prove.

Suppose that $n>0$. If $(x_i)_{i=1}^{2n}$ is minimal, there is nothing to prove.
Otherwise, one of the conditions (a),(b) of Definition \ref{def:minimal} is not
satisfied.

Suppose that (a) is not satisfied. Then there is $2\leq i\leq n$ such that
$(x_{2i}\ominus x_{2i-1})\wedge(x_{2i-1}\ominus x_{2i-2})=0$. Consider the
element $y=x_{2i-2}\oplus(x_{2i}\ominus x_{2i-1})$.
Clearly, $x_{2i-1}\vee y=x_{2i}$ and $x_{2i-1}\wedge y=x_{2i-2}$.
By elementary Boolean calculus, this implies
that $x_{2i}+x_{2i-1}=y+x_{2i-2}$ and
$$
x_{2i}+x_{2i-1}+x_{2i-2}+x_{2i-3}=y+x_{2i-2}+x_{2i-2}+x_{2i-3}=y+x_{2i-2}
$$.
Replacing the quadruple $(x_{2i},\dots,x_{2i-3})$ in $(x_i)_{i=1}^{2n}$
by $(y,x_{2i-2})$ shows that $x$ admits an $M$-chain representation
of length $2(n-1)$. By induction hypothesis, $x$ has a minimal
$M$-chain representation.

Suppose that (a) is satisfied and (b) is not. Similarly as in the
previous part of the proof, we can replace $(x_1,x_2)$ by 
$(0,x_2\ominus x_1)$, the resulting $M$-chain representation is then minimal.

\end{proof}
\begin{proposition}
Let $x\in B(M)$. Any two minimal $M$-chain representations of $x$ have the
same length.
\end{proposition}
\end{document}
